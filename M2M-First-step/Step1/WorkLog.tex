
\documentclass[10]{article}
\usepackage{graphicx}
\begin{document}
\title{Step 1 - Work Log \\
M2M Lectures\\
Grenoble University}
\author{Your Names Here}
\date{\today}
\maketitle

\section{Preface by Pr. Olivier Gruber}

This document is your work log for the first step in the 
M2M course, master-level, at the University of Grenoble, France.
You will have such a document for each step of our course
together.

This document has two parts. One part is about diverse
sections, each with a bunch of questions 
that you have to answers. The other part is really 
a laboratory log, keeping track of what you do, 
as you do.

The questions provide a guideline for your learning. 
They are not about getting a good grade if you answer them
correctly, they are about giving your pointers on what to 
learn about.

The goal of the questions is therefore not to be answered 
in three lines of text and be forgotten about. The questions
must be researched and thoroughly understood. Ask questions
around you if things are unclear, to your fellow students
and to me, your professor. 

Writing down the answers to the questions is a tool for helping
your learn and remember. Also, it keeps track of what you know,
the URLs you visited, the open questions that you are trouble with,
etc. The tools you used. It is intended to be a living document,
written as you go.

Ultimately, the goal of the document is to be kept for 
your personal records. If ever you will work on embedded
systems, trust me, you will be glad to have a written 
trace about all this.

{\bf REMEMBER:} plaggia is a crime that can get you evicted
forever from french universities... The solution is simple,
write using your own words or quote, giving the source of
the quoted text. Also, remember that you do not learn through
cut\&paste. You also do not learn much by watching somebody else
doing.

\section{Qemu}

\begin{enumerate}
\item 
What is Qemu for?
\item
Why cannot you run a linux kernel in a regular linux process?
\item
Comment the different options you used to start qemu.
\end{enumerate}

\section{Boot Process}

\begin{enumerate}
\item
How is an x86 machine booting up?
\item
Hints: Bios, MBR (Master Boot Record), Kernel.
\item
What is the role of each involved parts?
\item
How is built the disk image that you use to boot with qemu?
Describe its layout in terms of sectors and the contents of 
those sectors.
\end{enumerate}

\section{Using Eclipse to browse the sources} 

Explain how you configured Eclipse to be able to browse the given sources.

\section{Master Boot Record}

\begin{enumerate}
\item
From what sources (.c and .S files) is the MBR built?
\item
What is the purpose of those different files?
\item
What is an ELF? (Hint: man elf, Google is your friend) 
\item
Why is the objcopy program used? (Hint: look in the Makefile)
\item
What kind of information is available in an ELF file?
\item
Give the ELF layout of the MBR files (hint: readelf and objdump) 
\item
Look at the code in loader.c and understand it. 
\item
What are the function waitdisk, readsect, and readseg doing?
\item
Explain the dialog with the disk controller. (Hint: in/out functions).
\item
What can you say about the concepts at the software-hardware frontier?
\end{enumerate}

\section{Master Boot Record Debugging}

Use gdb to step through our bootloader.

Hints: 
\begin{enumerate}
\item 
Look at the dbg target in the Makefile.
\item
Look at the .gdbinit file.
\item
Use source layout in gdb.
\item
Use emacs as a front end.
\end{enumerate}

List and explain the various gdb commands you use.

\section{Our mini Kernel}

\begin{enumerate}
\item
What is the code in crt0.S doing?
\item
What are the function in/out for at this level?
\item
What are the inline attributes for?
\item
Explain why is your fan ramping up when you launch qemu with:

  \$ make clean ; make run

\item
Explain what is the relationship between the qemu option (-serial stdio)
and the COM1 concept in the program.
\item
Explain what is COM1 versus the console?
\end{enumerate}

\section{Debugging with Eclipse}

How did you setup your Eclipse to debug your mini kernel?

\section{Kernel Extensions}

{\bf IT IS MANDATORY TO USE THE DEBUGGER TO DEBUG YOUR CODE.}

\subsection{Echo on the screen}

This extension is to have the input from the UART be 
echoed on the console screen (the greenish output).
Do not forget that you have only 25 lines and you will
need to implement scrolling.

\subsection{History and line editing}

This extension is to have a history of typed lines.
A line is added to the history when the return key is pressed.
The arrow up and down allow you to scroll up and down in the history.
The arrow left and write allow you to move left and right in the current line.
The backspace and delete allow you delete characters.

\subsection{Echo on COM2}

This extension is to have the ability to have a printf-like 
capability on COM2. The code is in the kprintf.c file.

\noindent Hints: 
\begin{enumerate}
\item 
Look at the target run2 in the Makefile to know how
to setup COM2.
\item
Add the kprintf.c file to your kernel
\item
Launch with ``make run2'' and use a telnet connection
for COM2.
\end{enumerate}


\section{Laboratory Log}

\end{document}
